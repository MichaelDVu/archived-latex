\documentclass[12pt]{article}
\author{Michael Vu}
\title{Numerical Analysis HW 2}
\usepackage{pdfsync} %This package allows for some communication between your pdf viewer and your latex editor, so that you can double click on a line in the pdf, for instance, and it will jump to the corresponding line in your latex code.

\usepackage{amssymb, amsmath, graphicx,amsthm} %these are standard packages that I pretty much always use.

\usepackage{bm,stmaryrd,verbatim,color,amsbsy}
%\usepackage{showkeys}  %If you have trouble remembering what you labeled your equations, you can uncomment this package, and temporarily see the labels displayed.
\usepackage{epstopdf} %Since I use Maple sometimes to make figures, and I get it to output eps files, I use this package so that eps files can be inputted into my documents. 
%Mathematica has the ability to output figures as pdfs, which are handled automatically with the graphicx package enabled.

\usepackage[normalem]{ulem} %this is a package that lets you underline, but mostly I like the \sout command for striking out.

\usepackage{color} %this package lets us change the color of your font by writing something like {\color[rgb]{1,0,0} This would be red}.




\def\red #1{{\color[rgb]{1, 0,0}#1}} %To make things easy on myself, I define a simpler function like this, so I can write \red{This would be red} instead of the above babble.

\newtheorem{theorem}{Theorem}[section]
\newtheorem{lemma}[theorem]{Lemma}
\newtheorem{proposition}[theorem]{Proposition}
\newtheorem{corollary}[theorem]{Corollary} %These define theorem like environments.

\newenvironment{definition}[1][Definition]{\begin{trivlist}
\item[\hskip \labelsep {\bfseries #1}]}{\end{trivlist}}
\newenvironment{example}[1][Example]{\begin{trivlist}
\item[\hskip \labelsep {\bfseries #1}]}{\end{trivlist}}
\newenvironment{remark}[1][Remark]{\begin{trivlist}
\item[\hskip \labelsep {\bfseries #1}]}{\end{trivlist}} %These define various other types of math environments: definition, example, and remarks.

\newcommand{\ph}{\varphi}






\bibliographystyle{plain}


%Some common or not so common symbols that I wanted shortcuts for.
\newcommand{\eps}{\varepsilon}
\newcommand{\ddt}{\frac{\mbox{d}}{\mbox{d}t}}
\newcommand{\dA}{\,\mbox{d}A}
\newcommand{\dv}{\,\mbox{d}v}
\def\dx{\, \mbox{d} x}
\newcommand{\ds}{\,\mbox{d}s}
\newcommand{\dz}{\,\mbox{d}z}
\def\dzeta{\,\mbox{d}\zeta}
\newcommand{\dl}{\,\mbox{d}l}
\renewcommand{\div}{\mbox{ div}}
\newcommand{\grad}{\mbox{ grad }}




%bold letters usually used to denote vectors matrices.
\def\bfa{\mathbf{a}}
\def\bfA{\mathbf{A}}
\def\bfb{\mathbf{b}}
\def\bfB{\mathbf{B}}
\def\bfc{\mathbf{c}}
\def\bfC{\mathbf{C}}
\def\bfd{\mathbf{d}}
\def\bfD{\mathbf{D}}
\def\bfe{\mathbf{e}}
\def\bfE{\mathbf{E}}
\def\bff{\mathbf{f}}
\def\bfF{\mathbf{F}}
\def\bfg{\mathbf{g}}
\def\bfG{\mathbf{G}}
\def\bfh{\mathbf{h}}
\def\bfH{\mathbf{H}}
\def\bfi{\mathbf{i}}
\def\bfI{\mathbf{I}}
\def\bfj{\mathbf{j}}
\def\bfJ{\mathbf{J}}
\def\bfk{\mathbf{k}}
\def\bfK{\mathbf{K}}
\def\bfl{\mathbf{l}}
\def\bfL{\mathbf{L}}
\def\bfm{\mathbf{m}}
\def\bfM{\mathbf{M}}
\def\bfn{\mathbf{n}}
\def\bfN{\mathbf{N}}
\def\bfo{\mathbf{o}}
\def\bfO{\mathbf{O}}
\def\bfp{\mathbf{p}}
\def\bfP{\mathbf{P}}
\def\bfq{\mathbf{q}}
\def\bfQ{\mathbf{Q}}
\def\bfr{\mathbf{r}}
\def\bfR{\mathbf{R}}
\def\bfs{\mathbf{s}}
\def\bfS{\mathbf{S}}
\def\bft{\mathbf{t}}
\def\bfT{\mathbf{T}}
\def\bfu{\mathbf{u}}
\def\bfU{\mathbf{U}}
\def\bfv{\mathbf{v}}
\def\bfV{\mathbf{V}}
\def\bfw{\mathbf{w}}
\def\bfW{\mathbf{W}}
\def\bfx{\mathbf{x}}
\def\bfX{\mathbf{X}}
\def\bfy{\mathbf{y}}
\def\bfY{\mathbf{Y}}
\def\bfz{\mathbf{z}}
\def\bfZ{\mathbf{Z}}



%``Blackboard bold" letters \bbC is used for the complex plane, \bbN
\def\bba{\mathbb{a}}
\def\bbA{\mathbb{A}}
\def\bbb{\mathbb{b}}
\def\bbB{\mathbb{B}}
\def\bbc{\mathbb{c}}
\def\bbC{\mathbb{C}}
\def\bbd{\mathbb{d}}
\def\bbD{\mathbb{D}}
\def\bbe{\mathbb{e}}
\def\bbE{\mathbb{E}}
\def\bbf{\mathbb{f}}
\def\bbF{\mathbb{F}}
\def\bbg{\mathbb{g}}
\def\bbG{\mathbb{G}}
\def\bbh{\mathbb{h}}
\def\bbH{\mathbb{H}}
\def\bbi{\mathbb{i}}
\def\bbI{\mathbb{I}}
\def\bbj{\mathbb{j}}
\def\bbJ{\mathbb{J}}
\def\bbk{\mathbb{k}}
\def\bbK{\mathbb{K}}
\def\bbl{\mathbb{l}}
\def\bbL{\mathbb{L}}
\def\bbm{\mathbb{m}}
\def\bbM{\mathbb{M}}
\def\bbn{\mathbb{n}}
\def\bbN{\mathbb{N}}
\def\bbo{\mathbb{o}}
\def\bbO{\mathbb{O}}
\def\bbp{\mathbb{p}}
\def\bbP{\mathbb{P}}
\def\bbq{\mathbb{q}}
\def\bbQ{\mathbb{Q}}
\def\bbr{\mathbb{r}}
\def\bbR{\mathbb{R}}
\def\bbs{\mathbb{s}}
\def\bbS{\mathbb{S}}
\def\bbt{\mathbb{t}}
\def\bbT{\mathbb{T}}
\def\bbu{\mathbb{u}}
\def\bbU{\mathbb{U}}
\def\bbv{\mathbb{v}}
\def\bbV{\mathbb{V}}
\def\bbw{\mathbb{w}}
\def\bbW{\mathbb{W}}
\def\bbx{\mathbb{x}}
\def\bbX{\mathbb{X}}
\def\bby{\mathbb{y}}
\def\bbY{\mathbb{Y}}
\def\bbz{\mathbb{z}}
\def\bbZ{\mathbb{Z}}

\def\ca{\mathcal{a}}
\def\cA{\mathcal{A}}
\def\cb{\mathcal{b}}
\def\cB{\mathcal{B}}
\def\cc{\mathcal{c}}
\def\cC{\mathcal{C}}
\def\cd{\mathcal{d}}
\def\cD{\mathcal{D}}
\def\ce{\mathcal{e}}
\def\cE{\mathcal{E}}
\def\cf{\mathcal{f}}
\def\cF{\mathcal{F}}
\def\cg{\mathcal{g}}
\def\cG{\mathcal{G}}
\def\ch{\mathcal{h}}
\def\cH{\mathcal{H}}
\def\ci{\mathcal{i}}
\def\cI{\mathcal{I}}
\def\cj{\mathcal{j}}
\def\cJ{\mathcal{J}}
\def\ck{\mathcal{k}}
\def\cK{\mathcal{K}}
\def\cl{\mathcal{l}}
\def\cL{\mathcal{L}}
\def\cm{\mathcal{m}}
\def\cM{\mathcal{M}}
\def\cn{\mathcal{n}}
\def\cN{\mathcal{N}}
\def\co{\mathcal{o}}
\def\cO{\mathcal{O}}
\def\cp{\mathcal{p}}
\def\cP{\mathcal{P}}
\def\cq{\mathcal{q}}
\def\cQ{\mathcal{Q}}
\def\calr{\mathcal{r}}
\def\cR{\mathcal{R}}
\def\cs{\mathcal{s}}
\def\cS{\mathcal{S}}
\def\ct{\mathcal{t}}
\def\cT{\mathcal{T}}
\def\cu{\mathcal{u}}
\def\cU{\mathcal{U}}
\def\cv{\mathcal{v}}
\def\cV{\mathcal{V}}
\def\cw{\mathcal{w}}
\def\cW{\mathcal{W}}
\def\cx{\mathcal{x}}
\def\cX{\mathcal{X}}
\def\cy{\mathcal{y}}
\def\cY{\mathcal{Y}}
\def\cz{\mathcal{z}}
\def\cZ{\mathcal{Z}}

%Some operators that I wanted displayed in a similar font as you would want sin and cos functions (by the way, those can be called up by using \sin,\cos,\log,\ln, \tan,\exp,\sec,\csc,\sinh,\cosh,\tanh, there are probably more.
\DeclareMathOperator*{\sgn}{sgn}
\DeclareMathOperator*{\spn}{span}
\DeclareMathOperator*{\coker}{coker}
\DeclareMathOperator*{\Id}{Id}
\DeclareMathOperator*{\D}{D}
\def\RR{\mathbb{R}}

%more symbols that I wanted extra short commands for
\newcommand{\be}{\mathbf{e}}
\renewcommand{\L}{\Lambda}
\newcommand{\vs}{\varsigma}
\newcommand{\al}{\alpha}
\newcommand{\s}{\sigma}
\renewcommand{\k}{\kappa}
\renewcommand{\t}{\theta}
\newcommand{\w}{\omega}


%some bold greek letters.
\def\bfeta{\boldsymbol{\eta}}
\def\bfxi{\boldsymbol{\xi}}
\def\bfrho{\boldsymbol{\rho}}
\def\bfth{\boldsymbol{\t}}
\def\bflam{\boldsymbol{\lambda}}
\def\bfTh{\boldsymbol{\Theta}}
\def\bfnu{\boldsymbol{\nu}}

%some more commands
\newcommand{\drho}{\,{\rm d} \rho}
\newcommand{\dth}{\,{\rm d} \t}
\newcommand{\tilu}{\tilde{u}}
\newcommand{\vt}{\vartheta}



%All the stuff above is more or less the same for all my documents, of course this gives rise to vestigial parts, which should be eventually eliminated, but I'm sure people carry around pointless code for decades without noticing.

\begin{document} %This is where the document actually begins


\maketitle %Obviously this makes the title for you. It gathers the information I put at the top of the document (so it's easy to find and modify), like \author, \date, \title, etc.

\textbf{Problem 1.} In the case for determining the largest real value, Tom and Jim's codes produced drastically different results from one another. For my setup, I am running Windows 10, an AMD FX 8350 CPU, and my compiler is WinGW. Under these conditions, Tom's code where he multiplies the $x$ by 2 for each iteration, the largest output is approximately $9.22\times 10^{18}$. Whereas Jim's code where he adds $x$ to itself for each iteration, the largest output is over twice as large, approximating $1.70\times 10^{38}$. Jim's code also resulted in a ``Note: The following floating-point exceptions are signalling: IEEE\textunderscore OVERFLOW\textunderscore FLAG''. 

I decided to run both of their codes in Linux Mint 18 using the same hardware. Jim's code resulted in the same output as Windows 10/WinGW with the same message at the end of the program. However, Tom's code result was much different. Tom's code only output up to 8388608.00. Regardless of the operating system, both codes produced very different results from one another.

\bigskip

\textbf{Problem 2.} To explain the Tom and Jim situation is not something, I truly understand. My best guess is that it has to do with the number of digits being computed, rather than the actual value itself. For example, it would seem to me that storing the data $123456+123456$ would require more bits than $123456\times 2.0$, thus adding the number to itself automatically creates more space, which in turn, allows more digits.

\bigskip

\textbf{Problem 3.} After amending the codes from problem 1 for both Tom and Jim, I ran the programs and the results are as I predicted for Tom, but not what I expected for Jim. After setting the real variables to double precision, here is Tom's code:
\begin{verbatim}
program tom2
implicit none
double precision :: x
integer :: i

x = 1.0

do while (x + 1.0d0 /= x)
	print*, x
	x = x * 2.0d0
end do

stop
end program
\end{verbatim}

The output from this code results in the same amount of digits as the previous code, however, more digits are displayed with the modified code. His original code resulted with a max value of $9.22337204\times 10^{18}$ and the modified code resulted with a max value of $9.2233720368547758\times 10^{18}$, twice as many digits, but approximately the same value. When ran in Linux Mint 18, the original max value was 8388608.00, and the max value for the modified code was 4503599627370496.0, which is significantly larger with twice as many digits. Looking at Jim's modified code:

\begin{verbatim}
program jim2
implicit none
double precision :: x
integer :: i 

x = 1.0

do while (x + x /= x)
	print*, x
	x = x * 2.0d0
end do

stop
end
\end{verbatim}

The output from this code results in a value significantly larger than the original code. Jim's original code had a max value of $1.70141183\times 10^{38}$, and the modified code resulted in a max value of $8.9884656743115795\times 10^{307}$, over 8 times larger than the original max value. When ran in Linux Mint 18, Jim's modified code matched exactly as when ran in Windows 10.

\end{document}