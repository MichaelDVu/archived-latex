\documentclass[12pt]{article}
\usepackage{fancyhdr}
\usepackage{amsmath}
\usepackage{amssymb, amsmath, graphicx,amsthm,setspace}
\pagestyle{fancy}

\lhead[lh-even]{\textbf{Michael Vu}}  
\lfoot[lf-even]{} 
\chead[ch-even]{\textbf{1.26 Theorem}}  
\cfoot[cf-even]{} 
\rhead[rh-even]{September 12, 2017}  
\rfoot[rf-even]{}

\begin{document}
\noindent\textbf{1.26 Theorem.} (The Division Algorithm) Let $n,m\in\mathbb{N}$. Then (\textit{existence part}) there exist integers $q$ (for quotient) and $r$ (for remainder) such that 

\begin{equation*}
m=nq+r
\end{equation*}

\noindent and $0\leq r < n$...

\bigskip

\noindent\textbf{Proof.} Let $m,n\in\mathbb{N}$ be given. Consider $S=\{m-nq\mid m-nq\in\mathbb{N}$ and $0\leq m-nq\}$.

\bigskip

\noindent Note: Taking $q=0$, establishes $S$ as a non-empty set since $m\geq 0$.

\bigskip

\noindent By WOANN, $S$ contains a smallest element, $m-nq\in S$. Since $m-nq$ is minimal, $m-n(q+1) < 0$. Therefore, $m-nq < n$. Letting $r=m-nq$, $0\leq r < n$.\qed


\end{document}