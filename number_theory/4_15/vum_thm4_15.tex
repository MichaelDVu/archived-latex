\documentclass[12pt]{article}
\usepackage{fancyhdr}
\usepackage{amsmath}
\usepackage{amssymb, amsmath, graphicx,amsthm,setspace}
\pagestyle{fancy}

\lhead[lh-even]{\textbf{Michael Vu}}  
\lfoot[lf-even]{} 
\chead[ch-even]{\textbf{4.15 Theorem}}  
\cfoot[cf-even]{} 
\rhead[rh-even]{November 19, 2017}  
\rfoot[rf-even]{}

\begin{document}
\noindent\textbf{4.15 Theorem.} (Fermat's Little Theorem, Version I) If $p$ is a prime and $a$ is an integer relatively prime to $p$, then $a^{p-1}\equiv 1\pmod p$.

\bigskip

\noindent\textbf{Proof.} By Theorem 4.14, $a\cdot 2a\cdot 3a\cdots (p-1)a\equiv 1\cdot 2\cdot 3\cdot (p-1)\pmod p$. Simplifying,

\begin{equation*}
a^{p-1}(p-1)!\equiv (p-1)!\pmod p.
\end{equation*}

\noindent Notice since $p$ is prime, no number less than $p$ will divide $p$. $(p-1)!$ contains no factors of $p$. Thus, since $(p,(p-1)!)=1$, $a^{p-1}\equiv 1\pmod p$.\qed

\end{document}

