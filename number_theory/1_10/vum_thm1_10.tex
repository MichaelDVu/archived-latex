\documentclass[12pt]{article}
\usepackage{fancyhdr}
\usepackage{amsmath}
\usepackage{amssymb, amsmath, graphicx,amsthm,setspace}
\pagestyle{fancy}

\lhead[lh-even]{\textbf{Michael Vu}}  
\lfoot[lf-even]{} 
\chead[ch-even]{\textbf{1.10 Theorem}}  
\cfoot[cf-even]{} 
\rhead[rh-even]{August 30, 2017}  
\rfoot[rf-even]{}

\begin{document}
\noindent\textbf{1.10 Theorem.} Let $a,b,n\in\mathbb{Z}$ with $n>0$. If $a\equiv b\pmod{n}$, then $b\equiv a\pmod{n}$.

\bigskip

\noindent\textbf{Proof.} Let $a,b,n\in\mathbb{Z}$ with $n>0$ be given such that $a\equiv b\pmod{n}$. Then by definition, $n\mid(a-b)$. We may choose $k\in\mathbb{Z}$ such that $a-b=nk$. Multiplying both sides by $-1$,


\begin{align*}
-(a-b) &= -(nk), \\
b-a &= -kn.
\end{align*}


\noindent By CPI, we may choose $t\in\mathbb{Z}$ such that $-k=t$. Therefore, $b-a=tn$, and by definition of divisibility, $n\mid(b-a)$. Lastly, by definition of congruence of modulo, $b\equiv a\pmod{n}$.\qed



\end{document}