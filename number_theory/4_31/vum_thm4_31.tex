\documentclass[12pt]{article}
\usepackage{fancyhdr}
\usepackage{amsmath}
\usepackage{amssymb, amsmath, graphicx,amsthm,setspace}
\pagestyle{fancy}

\lhead[lh-even]{\textbf{Michael Vu}}  
\lfoot[lf-even]{} 
\chead[ch-even]{\textbf{4.31 Theorem}}  
\cfoot[cf-even]{} 
\rhead[rh-even]{November 26, 2017}  
\rfoot[rf-even]{}

\begin{document}
\noindent\textbf{4.31 Theorem.} Let $n$ be a natural number and let $x_1,x_2,...x_{\phi(n)}$ be the distinct natural numbers less than or equal to $n$ that are relatively prime to $n$. Let $a$ be a non-zero integer relatively prime to $n$ and let $i$ and $j$ be different natural numbers less than or equal to $\phi(n)$. Then $ax_i\not\equiv ax_j\pmod n$.

\bigskip

\noindent\textbf{Proof.} Suppose not. That is, suppose $ax_i\equiv ax_j\pmod n$. Since $(a,n)=1$, by Theorem 1.45, $x_i\equiv x_j\pmod n$. Let $X=\{x_1,x_2,...x_{\phi(n)}\}$ be a subset of the CCRS modulo $n$. By Theorem 3.17, no two elements of $X$ are congruent modulo $n$, which contradicts $x_i\equiv x_j\pmod n$. Thus, $ax_i\not\equiv ax_j\pmod n$.\qed 

\end{document}

