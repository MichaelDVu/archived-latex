\documentclass[12pt]{article}
\usepackage{fancyhdr}
\usepackage{amsmath}
\usepackage{amssymb, amsmath, graphicx,amsthm,setspace}
\pagestyle{fancy}

\lhead[lh-even]{\textbf{Michael Vu}}  
\lfoot[lf-even]{} 
\chead[ch-even]{\textbf{1.11 Theorem}}  
\cfoot[cf-even]{} 
\rhead[rh-even]{August 30, 2017}  
\rfoot[rf-even]{}

\begin{document}
\noindent\textbf{1.11 Theorem.} Let $a,b,c,n\in\mathbb{Z}$ with $n>0$. If $a\equiv b\pmod{n}$ and $b\equiv c\pmod{n}$, then $a\equiv c\pmod{n}$.

\bigskip

\noindent\textbf{Proof.} Let $a,b,c,n\in\mathbb{Z}$ with $n>0$ be given such that $a\equiv b\pmod{n}$ and $b\equiv c\pmod{n}$. Then by definition, $n\mid(a-b)$ and $n\mid(b-c)$. We may choose $t,u\in\mathbb{Z}$ such that $a-b=nt$ and $b-c=nu$, by definition of divisibility. Using algebra, $b=nu+c$, and by substitution,


\begin{align*}
a-(nu+c) &= nt, \\
a-nu-c &= nt, \\
a-c &= nt+nu \\
&= n(t+u).
\end{align*}

\noindent By CPI, we may choose $k\in\mathbb{Z}$ such that $t+u=k$. Therefore, $a-c=nk$, and by definition of divisibility, $n\mid(a-c)$. Lastly, by definition of congruence of modulo, $a\equiv c\pmod{n}$.\qed



\end{document}