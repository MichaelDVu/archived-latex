\documentclass[12pt]{article}
\usepackage{fancyhdr}
\usepackage{amsmath}
\usepackage{amssymb, amsmath, graphicx,amsthm,setspace}
\pagestyle{fancy}

\lhead[lh-even]{\textbf{Michael Vu}}  
\lfoot[lf-even]{} 
\chead[ch-even]{\textbf{4.2 Theorem}}  
\cfoot[cf-even]{} 
\rhead[rh-even]{November 7, 2017}  
\rfoot[rf-even]{}

\begin{document}
\noindent\textbf{4.2 Theorem.} Let $a,n\in\mathbb{N}$ with $(a,n)=1$. Then $(a^j,n)=1$ for any $j\in\mathbb{N}$.

\bigskip

\noindent\textbf{Proof.} Let $j=2$ be the base case such that $(a^2,n)=1$. Since $(a,n)=1$, by Theorem 1.38, there exists $x,y\in\mathbb{Z}$ such that $ax+ny=1$. Multiplying by $ax+ny$,

\begin{align*}
ax+ny &= (ax+ny)(ax+ny)\\
&= a^2x^2 +2axny +n^2y^2\\
&= a^2x^2 + n(2axy+ny^2)\\
&= a^2x' + ny'
\end{align*}

\noindent where, by CPI, integers $x'=x^2$ and $y'=2axy+ny^2$. Looking at the left hand side, we know that $ax+ny=1$. Thus,

\begin{equation*}
1=a^2x'+ny'.
\end{equation*}

\noindent By Theorem 1.39, $(a^2,n)=1$. Thus, the base case is true. Suppose our assumption is true for all $j$ where $1 \leq j \leq k$. By induction, we want to show $(a^{k+1},n)=1$ is also true. By our assumption, $(a^k,n)=1$. By Theorem 1.38, there exists $t,u\in\mathbb{Z}$ such that $a^kt+nu=1$. Multiplying by $ax+ny$,

\begin{align*}
ax+ny &= (a^kt+nu)(ax+ny)\\
&= a^{k+1}tx +a^ktny + axnu +n^2uy\\
&= a^{k+1}tx + n(a^kty+axu+nuy)\\
&= a^{k+1}t' + nu'
\end{align*}

\noindent where, by CPI, integers $t'=tx$ and $u'=a^kty+axu+nuy$. Looking at the left hand side, we know that $ax+ny=1$. Thus,

\begin{equation*}
1=a^{k+1}t' + nu'.
\end{equation*}

\noindent By Theorem 1.39, $(a^{k+1},n)=1$. Thus, $(a^j,n)=1$ for any $j\in\mathbb{N}$.\qed

\end{document}

