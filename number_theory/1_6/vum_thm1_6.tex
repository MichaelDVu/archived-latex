\documentclass[12pt]{article}
\usepackage{fancyhdr}
\usepackage{amsmath}
\usepackage{amssymb, amsmath, graphicx,amsthm,setspace}
\pagestyle{fancy}

\lhead[lh-even]{\textbf{Michael Vu}}  
\lfoot[lf-even]{} 
\chead[ch-even]{\textbf{1.6 Theorem}}  
\cfoot[cf-even]{} 
\rhead[rh-even]{August 30, 2017}  
\rfoot[rf-even]{}

\begin{document}
\noindent\textbf{1.6 Theorem.} Let $a,b,c\in\mathbb{Z}$. If $a\mid b$,then $a\mid bc$.

\bigskip

\noindent\textbf{Proof.} Let $a,b,c\in\mathbb{Z}$ be given such that $a\mid b$. We may choose $k\in\mathbb{Z}$ such that $b=ka$. Multiplying both sides by $c$,


\begin{align*}
b(c) &= ka(c) \\
&= a(kc).
\end{align*}


\noindent By CPI, we may choose $m\in\mathbb{Z}$ such that $kc=m$. Therefore, $bc=am$, and by definition, $a\mid(bc)$.\qed



\end{document}