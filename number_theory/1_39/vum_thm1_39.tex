\documentclass[12pt]{article}
\usepackage{fancyhdr}
\usepackage{amsmath}
\usepackage{amssymb, amsmath, graphicx,amsthm,setspace}
\pagestyle{fancy}

\lhead[lh-even]{\textbf{Michael Vu}}  
\lfoot[lf-even]{} 
\chead[ch-even]{\textbf{1.39 Theorem}}  
\cfoot[cf-even]{} 
\rhead[rh-even]{September 14, 2017}  
\rfoot[rf-even]{}

\begin{document}
\noindent\textbf{1.39 Theorem.} Let $a,b\in\mathbb{Z}$. If there exist $x,y\in\mathbb{Z}$ with $ax+by=1$, then $(a,b)=1$.

\bigskip
\noindent\textbf{Proof.} Let $a,b\in\mathbb{Z}$ be given. Let $x,y\in\mathbb{Z}$ such that $ax+by=1$. We want to show $(a,b)=1$. Let $d=(a,b)$. It follows that $d\mid a$ and $d\mid b$. Observing $ax+by=1$, since $d\mid a$, $d$ divides any multiple of $a$. Similarly, since $d\mid b$, $d$ divides any multiple of $b$. Both terms are divisible by $d$, therefore, the sum is divisible my $d$. Thus, $d\mid 1$.

\bigskip

\noindent Clearly, the only two numbers that divide 1, are -1 and 1. Since $d\geq 1$, we could assume $d=1$. But suppose not. That is, given $d|1$, suppose $d>1$. By definition, $1=dt$ for $t\in\mathbb{Z}$. Since $d>1$, it follows that $1\leq t < dt$. Thus, $1<dt$. This contradicts the definition of $d|1$. Thus, $d=1=(a,b)$.\qed

\end{document}