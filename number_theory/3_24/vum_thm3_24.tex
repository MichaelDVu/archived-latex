\documentclass[12pt]{article}
\usepackage{fancyhdr}
\usepackage{amsmath}
\usepackage{amssymb, amsmath, graphicx,amsthm,setspace}
\pagestyle{fancy}

\lhead[lh-even]{\textbf{Michael Vu}}  
\lfoot[lf-even]{} 
\chead[ch-even]{\textbf{3.24 Theorem}}  
\cfoot[cf-even]{} 
\rhead[rh-even]{October 31, 2017}  
\rfoot[rf-even]{}

\begin{document}
\noindent\textbf{3.24 Theorem.} Let $a,b,n\in\mathbb{Z}$ with $n>0$. Then

\bigskip

\noindent 1. The congruence $ax\equiv m\pmod n$ is solvable in integers if and only if $(a,n)|b$;

\bigskip

\noindent 2. If $x_0$ is a solution to the congruence $ax\equiv b\pmod n$, then all solutions are given by 

\begin{equation*}
x_0+\left(\frac{n}{(a,n)}m\right)\pmod n
\end{equation*}

\noindent for $m=0,1,2,...,(a,n)-1$; and

\bigskip

\noindent 3. If $ax\equiv b\pmod n$ has a solution, then there are exactly $(a,n)$ solutions in the canonical complete residue system modulo $n$.

\bigskip

\noindent\textbf{Proof.} 1. This is exactly Theorem 3.20.

\bigskip

\noindent 2. Let $x=x_0$ be an integer solution to $ax\equiv b\pmod n$. By Theorem 1.53, all solutions are given by $x=x_0+\frac{nk}{(a,n)}$ for some $k\in\mathbb{Z}$. We want to show $x_0+\frac{nk}{(a,n)}\equiv x_0+\frac{nm}{(a,n)} \pmod n$ for some $m=1,2,...,(a,n)-1$. Applying the division algorithm on $k$ by $(a,n)$ gives $k=(a,n)q+m$. Thus,

\begin{align*}
x_0+\frac{nk}{(a,n)} &\equiv x_0+\frac{n[(a,n)q+m]}{(a,n)}\\
&\equiv x_0 + nq + \frac{nm}{(a,n)} \text{, and since } nq\equiv 0\pmod n, \\
&\equiv x_0 + \frac{nm}{(a,n)}\pmod n.
\end{align*}

\noindent Thus, all solutions are given by $x_0+\left(\frac{n}{(a,n)}m\right)\pmod n$ for \\
$m=0,1,2,...,(a,n)-1$.

\bigskip

\noindent 3. Let $ax\equiv b\pmod n$ have a solution. We want to show there are exactly $(a,n)$ solutions in the canonical complete residue system modulo $n$. Suppose not. That is, let $x_0+\frac{nm}{(a,n)}\equiv x_0+\frac{nk}{(a,n)} \pmod n$ where $k\not= m$ and\\
 $0\leq k < m \leq (a,n)-1$. Thus,

\begin{align*}
x_0+\frac{nm}{(a,n)} &\equiv x_0+\frac{nk}{(a,n)} \pmod n,\\
\frac{nm}{(a,n)} &\equiv \frac{nk}{(a,n)} \pmod n.
\end{align*}

\noindent This implies

\begin{align*}
n &| \left[\frac{nk}{(a,n)}-\frac{nk}{(a,n)}\right]\\
n &| \frac{n}{(a,n)}(m-k).
\end{align*}

\noindent Some things to observe, 

\begin{equation*}
0\leq k < m \leq (a,n)-1 < (a,n) \leq n.
\end{equation*}

\noindent One of the properties of divisibility says for $n$ to divide $\frac{n}{(a,n)}(m-k)>0$, the condition $n\leq \frac{n}{(a,n)}(m-k)$ must occur. Observing the above inequality, we find 

\begin{equation*}
m-k \leq (a,n)-1.
\end{equation*}

\noindent Thus,

\begin{align*}
\frac{n}{(a,n)}(n-k) \leq \frac{n}{(a,n)}[(a,n)-1] &= n-\frac{1}{(a,n)} < n.
\end{align*}

\noindent Since $n-\frac{n}{(a,n)}$ is less than $n$, we have a contradiction. Thus, if $ax\equiv b\pmod n$ has a solution, then there are exactly $(a,n)$ solutions in the canonical complete residue system modulo $n$.\qed

\end{document}

