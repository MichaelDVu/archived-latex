\documentclass[12pt]{article}
\usepackage{fancyhdr}
\usepackage{amsmath}
\usepackage{amssymb, amsmath, graphicx,amsthm,setspace}
\pagestyle{fancy}

\lhead[lh-even]{\textbf{Michael Vu}}  
\lfoot[lf-even]{} 
\chead[ch-even]{\textbf{4.16 Theorem}}  
\cfoot[cf-even]{} 
\rhead[rh-even]{November 19, 2017}  
\rfoot[rf-even]{}

\begin{document}
\noindent\textbf{4.16 Theorem.} (Fermat's Little Theorem, Version II) If $p$ is a prime and $a$ is any integer, then $a^{p}\equiv a\pmod p$.

\bigskip

\noindent\textbf{Proof.} We will examine by cases.

\bigskip

\noindent\textbf{Case 1.} Suppose $(a,p)=1$. By Theorem 4.15, $a^{p-1}\equiv 1\pmod p$. Multiplying $a$ to both sides,

\begin{align*}
a^{p-1}a &\equiv 1(a)\pmod p,\\
a^p &\equiv a\pmod p.
\end{align*}

\noindent Thus, letting $(a,p)=1$ satisfies the theorem.

\bigskip

\noindent\textbf{Case 2.} Suppose $(a,p)\not= 1$. This implies $(a,p)=p$. Since $p|a$, $a\equiv 0\pmod p$. Also, $p$ divides any integer multiple of $a$ let's us know $a^p\equiv 0\pmod p$. Thus, $a^p\equiv a\pmod p$ by transitivity and reflexivity.

\bigskip

\noindent Since both cases are true, if $p$ is a prime and $a$ is any integer, then $a^{p}\equiv a\pmod p$.\qed

\end{document}

