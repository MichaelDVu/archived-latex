\documentclass[12pt]{article}
\usepackage{fancyhdr}
\usepackage{amsmath}
\usepackage{amssymb, amsmath, graphicx,amsthm,setspace}
\pagestyle{fancy}

\lhead[lh-even]{\textbf{Michael Vu}}  
\lfoot[lf-even]{} 
\chead[ch-even]{\textbf{4.14 Theorem}}  
\cfoot[cf-even]{} 
\rhead[rh-even]{November 19, 2017}  
\rfoot[rf-even]{}

\begin{document}
\noindent\textbf{4.14 Theorem.} Let $p$ be a prime and let $a$ be an integer not divisible by $p$; that is, $(a,p)=1$. Then 

\begin{equation*}
a\cdot 2a\cdot 3a\cdots (p-1)a\equiv 1\cdot 2\cdot 3\cdot (p-1)\pmod p. 
\end{equation*}

\bigskip

\noindent\textbf{Proof.} Let $A=\{a,2a,3a,...,(p-1)a\}$ and $N=\{1,2,3,...,p-1\}$. By Theorem 4.13, $ia\equiv j\pmod p$ where $1\leq i,j \leq p-1$. Note that $p|pa$ is equivalent to $pa\equiv 0\pmod p$. By Theorem 1.14, the product of $A$ is congruent to the product of $N$ modulo $p$. Thus, $a\cdot 2a\cdot 3a\cdots (p-1)a\equiv 1\cdot 2\cdot 3\cdot (p-1)\pmod p$.\qed

\end{document}

