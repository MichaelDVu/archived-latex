\documentclass[12pt]{article}
\usepackage{fancyhdr}
\usepackage{amsmath}
\usepackage{amssymb, amsmath, graphicx,amsthm,setspace}
\pagestyle{fancy}

\lhead[lh-even]{\textbf{Michael Vu}}  
\lfoot[lf-even]{} 
\chead[ch-even]{\textbf{1.18 Theorem}}  
\cfoot[cf-even]{} 
\rhead[rh-even]{September 9, 2017}  
\rfoot[rf-even]{}

\begin{document}
\noindent\textbf{1.18 Theorem.} Let $a,b,k,n\in\mathbb{Z}$ with $k,n>0$. If $a\equiv b\pmod{n}$, then $a^k\equiv b^k\pmod{n}$.

\bigskip

\noindent\textbf{Proof.} Let $a,b,k,n\in\mathbb{Z}$ with $k,n>0$ be given such that $a\equiv b\pmod{n}$. Consider the base case where $k=1$. By definition, $a-b=nx$ for some $x\in\mathbb{Z}$, which proves the base case.

\bigskip

\noindent By induction, we want to show $a^{t+1}\equiv b^{t+1}\pmod{n}$, provided $a^t\equiv b^t\pmod{n}$ for some $t\in\mathbb{Z}$. By definition, $a-b=nx$ (remembering $a=nx+b$) and $a^t-b^t=ny$ for some $x,y\in\mathbb{Z}$.

\begin{align*}
a^{t+1}-b^{t+1} &= aa^t - bb^t\\
&= (nx+b)a^t - bb^t\\
&= nxa^t + ba^t - bb^t\\
&= nxa^t + b(a^t-b^t)\\
&= nxa^t + b(ny)\\
&= n(xa^t + by).
\end{align*}

\noindent By CPI, $xa^t + by=z$ for some $z\in\mathbb{Z}$. Thus, $a^{t+1}\equiv b^{t+1}\pmod{n}$, and $a^k\equiv b^k\pmod{n}$.\qed

\end{document}