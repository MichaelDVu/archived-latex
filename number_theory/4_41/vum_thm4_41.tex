\documentclass[12pt]{article}
\usepackage{fancyhdr}
\usepackage{amsmath}
\usepackage{amssymb, amsmath, graphicx,amsthm,setspace}
\pagestyle{fancy}

\lhead[lh-even]{\textbf{Michael Vu}}  
\lfoot[lf-even]{} 
\chead[ch-even]{\textbf{4.41 Theorem}}  
\cfoot[cf-even]{} 
\rhead[rh-even]{November 28, 2017}  
\rfoot[rf-even]{}

\begin{document}
\noindent\textbf{4.41 Theorem.} (Wilson's Theorem) If $p$ is a prime, then $(p-1)! \equiv -1\pmod p$.

\bigskip

\noindent\textbf{Proof.} Let a prime $p$ be given. Letting $p=2$, 

\begin{align*}
(2-1)! &\equiv -1\pmod 2, \\
1 &\equiv -1\pmod 2.
\end{align*}

\noindent We find $(p-1)! \equiv -1\pmod p$ to be true when $p=2$. Note $p=3$ is also trivial. Suppose $p>3$. By Theorem 4.40, $(p-2)!\equiv 1\pmod p$. By definition,

\begin{align*}
pk &= (p-2)!-1 \text{ for some } k\in\mathbb{Z}.
\end{align*}

\noindent Multiplying both sides by $p-1$,

\begin{align*}
pk(p-1) &= (p-1)(p-2)!-(p-1) \\
&= (p-1)! - p + 1.
\end{align*}

\noindent Rearranging,

\begin{align*}
(p-1)!+1 &= pk(p-1)+p \\
&= ppk-pk+p\\
&= p(pk-k+1).
\end{align*}

\noindent By CPI, let $pk-k+1=k'\in\mathbb{Z}$ such that $(p-1)!+1=pk'$. Thus, by definition, $(p-1)! \equiv -1\pmod p$.\qed

\end{document}

