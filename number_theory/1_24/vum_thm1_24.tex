\documentclass[12pt]{article}
\usepackage{fancyhdr}
\usepackage{amsmath}
\usepackage{amssymb, amsmath, graphicx,amsthm,setspace}
\pagestyle{fancy}

\lhead[lh-even]{\textbf{Michael Vu}}  
\lfoot[lf-even]{} 
\chead[ch-even]{\textbf{1.24 Theorem}}  
\cfoot[cf-even]{} 
\rhead[rh-even]{September 9, 2017}  
\rfoot[rf-even]{}

\begin{document}
\noindent\textbf{1.24 Theorem.} An integer is divisible by 5 if and only if its last digit is 0 or 5.

\bigskip

\noindent\textbf{Proof.} Let $n\in\mathbb{Z}$ be divisible by 5 such that $n=5k$ for some $k\in\mathbb{Z}$. If $k$ is even then $k=2p$ for some $p\in\mathbb{Z}$.
 
\begin{align*}
n &= 5(2p)\\
&= 10p.
\end{align*}

\noindent Observe all multiples of 10 end in 0.

\bigskip

\noindent If $k$ is odd, then $k=2p+1$.

\begin{align*}
n &= 5(2p+1)\\
&= 10p + 5.
\end{align*}

\noindent Any multiple of 10 plus 5 ends in 5.

\bigskip

\noindent Suppose now we let $n$ be a number whose last digit is 0 or 5. Any number whose last digit is 0 can be represented as $10q$ for some $q\in\mathbb{Z}$. Since 10 is divisible by 5, 5 divides any multiple of 10. Any number whose last digit is 5 can be represented as $10q+5$. Since 5 divides any multiple of $10q$ and 5, 5 divides $10q+5$. Thus, an integer is divisible by 5 if and only if its last digit is 0 or 5.\qed

\end{document}