\documentclass[12pt]{article}
\usepackage{fancyhdr}
\usepackage{amsmath}
\usepackage{amssymb, amsmath, graphicx,amsthm,setspace}
\pagestyle{fancy}

\lhead[lh-even]{\textbf{Michael Vu}}  
\lfoot[lf-even]{} 
\chead[ch-even]{\textbf{3.28 Theorem}}  
\cfoot[cf-even]{} 
\rhead[rh-even]{November 4, 2017}  
\rfoot[rf-even]{}

\begin{document}
\noindent\textbf{3.28 Theorem.} Let $a,b,m,n\in\mathbb{Z}$ with $m,n>0$ and $(m,n)=1$. Then the system

\begin{align*}
x &\equiv a\pmod n\\
x &\equiv b\pmod m
\end{align*}

\noindent has a solution unique modulo $mn$.

\bigskip

\noindent\textbf{Proof.} Suppose

\begin{align*}
x &\equiv a\pmod n\\
x &\equiv b\pmod m
\end{align*}

\noindent and 

\begin{align*}
y &\equiv a\pmod n\\
y &\equiv b\pmod m.
\end{align*}

\noindent By Theorem 1.13,

\begin{align*}
x - y &\equiv a - a\pmod n & x - y &\equiv b - b\pmod m, \\
x - y &\equiv 0\pmod n & x - y &\equiv 0\pmod m.
\end{align*}

\noindent Now our system is 

\begin{align*}
x - y &\equiv 0\pmod n \\
x - y &\equiv 0\pmod m.
\end{align*}

\noindent This implies, $n|(x-y)$ and $m|(x-y)$. By Theorem 1.42, $mn|(x-y)$, implying $x\equiv y\pmod {mn}$. Since $(m,n)=1$, we can conclude the system has a solution by Theorem 3.27. Furthermore, the solution is unique modulo $mn$.\qed

\end{document}

