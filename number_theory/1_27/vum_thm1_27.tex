\documentclass[12pt]{article}
\usepackage{fancyhdr}
\usepackage{amsmath}
\usepackage{amssymb, amsmath, graphicx,amsthm,setspace}
\pagestyle{fancy}

\lhead[lh-even]{\textbf{Michael Vu}}  
\lfoot[lf-even]{} 
\chead[ch-even]{\textbf{1.27 Theorem}}  
\cfoot[cf-even]{} 
\rhead[rh-even]{September 12, 2017}  
\rfoot[rf-even]{}

\begin{document}
\noindent\textbf{1.27 Theorem.} (The Division Algorithm) \textit{(continued from 1.26)}...Moreover (\textit{uniqueness part}), if $q,q'$ and $r,r'$ are any integers that satisfy

\begin{align*}
m&=nq+r \\
&=nq'+r'
\end{align*} 

\noindent with $0\leq r,r' <n$, then $q=q'$ and $r=r'$.

\bigskip

\noindent\textbf{Proof.} Let 

\begin{align*}
m&=nq+r \\
&=nq'+r'
\end{align*}

\noindent with $0\leq r,r' <n$ be given.

\begin{align*}
nq+r &= nq'+r', \\
nq-nq' &= r'-r, \\
n(q-q') &= r'-r.
\end{align*}

\noindent Since $0\leq r,r' <n$, this implies $-n < r'-r < n$. Substituting,

\begin{align*}
-n &< n(q-q') < n, \\
 -1 &< q-q' < 1.
\end{align*}

\noindent Therefore $q-q'=0$, and $q=q'$. From here its easy to see, using substitution

\begin{align*}
nq'+r &= nq'+r', \\
nq'-nq'+r &= r', \\
r &= r'.
\end{align*}

\noindent Since $q=q'$ and $r=r'$, there is uniqueness.\qed

\end{document}