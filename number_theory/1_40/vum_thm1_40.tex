\documentclass[12pt]{article}
\usepackage{fancyhdr}
\usepackage{amsmath}
\usepackage{amssymb, amsmath, graphicx,amsthm,setspace}
\pagestyle{fancy}

\lhead[lh-even]{\textbf{Michael Vu}}  
\lfoot[lf-even]{} 
\chead[ch-even]{\textbf{1.40 Theorem}}  
\cfoot[cf-even]{} 
\rhead[rh-even]{September 19, 2017}  
\rfoot[rf-even]{}

\begin{document}
\noindent\textbf{1.40 Theorem.} For any integers $a$ and $b$ not both 0, there are integers $x$ and $y$ such that $ax+by=(a,b)$.

\bigskip

\noindent\textbf{Proof.} Let $d=(a,b)$ and $ax+by=k$ for $k\in\mathbb{N}$. Since $d|a$ and $d|b$, $d|k$. Thus, $d\leq k$. Let $S= \{\text{all }c \text{ that can be written as }ax+by \mid c\in\mathbb{N}\}$. Letting $x=a$ and $b=y$, we find that $a^2+b^2$ equals a natural number. Thus, the set is non-empty. By the WOANN, there exists a smallest element, call it $k$. Suppose $k$ does not divide $a$. By TDA,

\begin{align*}
a &= kq + r,\\
r &= a - kq  \text{ with } 0< r < k.
\end{align*}

\noindent Substituting $k=ax+by$ into $r$, 

\begin{align*}
r &= a - q(ax+by) \\
&= a - aqx + bqy\\
&= a(1-qx) + b(qy).
\end{align*}

\noindent Notice now that $r$ can be written as $ax'+by'$ which contradicts $k$ being the smallest that can be expressed in that form. Thus, $k|a$. Without loss of generality, the same argument can be such that $k|b$. Thus, $k=(a,b)=d$.

\bigskip

\noindent Gathering our info, we have $(a,b)\leq k$ and $(a,b)=k$. Since $k$ cannot be greater than AND equal to $(a,b)$, it must be that $k=(a,b)$. Thus, $ax+by=(a,b)$.\qed

\end{document}