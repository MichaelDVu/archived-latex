\documentclass[12pt]{article}
\usepackage{fancyhdr}
\usepackage{amsmath}
\usepackage{amssymb, amsmath, graphicx,amsthm,setspace}
\pagestyle{fancy}

\lhead[lh-even]{\textbf{Michael Vu}}  
\lfoot[lf-even]{} 
\chead[ch-even]{\textbf{Lemma IR}}  
\cfoot[cf-even]{} 
\rhead[rh-even]{October 7, 2017}  
\rfoot[rf-even]{}

\begin{document}
\noindent\textbf{Lemma IR} For all primes $p$, $\sqrt{p}$ is irrational.

\bigskip

\noindent\textbf{Proof.} Suppose not. That is let $\sqrt{p}\in\mathbb{Q}$. By definition, $\sqrt{p}=\frac{a}{b}$ for $a,b\in\mathbb{Z}$. Suppose $\frac{a}{b}$ is irreducible such that $(a,b)=1$. Squaring both sides, 

\begin{align*}
p &= \frac{a^2}{b^2}, \\
pb^2 &= a^2.
\end{align*}

\noindent Since $p|a^2$, $p$ must be a prime factor of $a$. Thus $a=pk$ for some $k\in\mathbb{Z}$. Substituting,

\begin{align*}
pb^2 &= (pk)^2, \\
b &= pk^2.
\end{align*}

\noindent Notice that both $a$ and $b$ share the prime factor $p$. Since primes are greater than 1, this contradictions the assumption that $(a,b)=1$. Thus, for all primes $p$, $\sqrt{p}$ is irrational.\qed



\end{document}