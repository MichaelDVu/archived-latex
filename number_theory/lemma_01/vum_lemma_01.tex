\documentclass[12pt]{article}
\usepackage{fancyhdr}
\usepackage{amsmath}
\usepackage{amssymb, amsmath, graphicx,amsthm,setspace}
\pagestyle{fancy}

\lhead[lh-even]{\textbf{Michael Vu}}  
\lfoot[lf-even]{} 
\chead[ch-even]{\textbf{Lemma 01}}  
\cfoot[cf-even]{} 
\rhead[rh-even]{September 19, 2017}  
\rfoot[rf-even]{}

\begin{document}
\noindent\textbf{Lemma 01.} Let $a,b\in\mathbb{Z}$ and both not 0. If $(a,b)=d$, then for $a=da'$ and $b=db'$, $(a',b')=1$.

\bigskip

\noindent\textbf{Proof.} Let $a,b\in\mathbb{Z}$ with both not 0 and $(a,b)=d$ be given such that $a=da'$ and $b=db'$. By contradiction, suppose $(a',b')\not= 1$. Thus, $(a',b')=k$ such that $a'=ka''$ and $b'=kb''$. Substituting in to $a$ and $b$, 

\begin{equation*}
a=dka'' \text{ and } b=dkb''
\end{equation*}

\noindent Letting $dk=t$ for $t\in\mathbb{Z}$,

\begin{equation*}
a=ta'' \text{ and } b=tb''.
\end{equation*}

\noindent Notice that $t|a$ and $t|b$. Also observe that $t>d$. This contradicts $(a,b)=d$. Thus, $(a',b')=1$.\qed

\end{document}