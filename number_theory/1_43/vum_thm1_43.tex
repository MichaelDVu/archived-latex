\documentclass[12pt]{article}
\usepackage{fancyhdr}
\usepackage{amsmath}
\usepackage{amssymb, amsmath, graphicx,amsthm,setspace}
\pagestyle{fancy}

\lhead[lh-even]{\textbf{Michael Vu}}  
\lfoot[lf-even]{} 
\chead[ch-even]{\textbf{1.43 Theorem}}  
\cfoot[cf-even]{} 
\rhead[rh-even]{September 24, 2017}  
\rfoot[rf-even]{}

\begin{document}
\noindent\textbf{1.43 Theorem.} Let $a,b,n\in\mathbb{Z}$. If $(a,n)=1$ and $(b,n)=1$, then $(ab,n)=1$.

\bigskip

\noindent\textbf{Proof.} Let $a,b,n\in\mathbb{Z}$ be given such that $(a,n)=1$ and $(b,n)=1$. By Theorem 1.38, since $(a,n)=1$ and $(b,n)=1$, there exists $x,y,s,t\in\mathbb{Z}$ such that $ax+ny=1$ and $bs+nt=1$, respectively. Multiplying the two equations together,

\begin{align*}
1 &= (ax+ny)(bs+nt) \\
&= axbs + axnt + nybs + n^2yt\\
&= ab(xs) + n(axt + ybs + nyt).
\end{align*}

\noindent By CPI, $xy=q$ and $axt + ybs + nyt=r$ for some $q,r\in\mathbb{Z}$. Thus, $abq + nr = 1$. By Theorem 1.39, $(ab,n)=1$.\qed

\end{document}