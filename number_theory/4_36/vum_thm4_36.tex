\documentclass[12pt]{article}
\usepackage{fancyhdr}
\usepackage{amsmath}
\usepackage{amssymb, amsmath, graphicx,amsthm,setspace}
\pagestyle{fancy}

\lhead[lh-even]{\textbf{Michael Vu}}  
\lfoot[lf-even]{} 
\chead[ch-even]{\textbf{4.36 Theorem}}  
\cfoot[cf-even]{} 
\rhead[rh-even]{December 02, 2017}  
\rfoot[rf-even]{}

\begin{document}
\noindent\textbf{4.36 Theorem.} Let $p$ be a prime and $a\in\mathbb{Z}$ where $1\leq a <p$. Then there exists a unique integer $b< p$ such that $ab\equiv 1\pmod p$.

\bigskip

\noindent\textbf{Proof.} Since $(a,p)=1$, by Theorem 1.38, there exists $x,y\in\mathbb{Z}$ such that $ax+py=1$. Applying TDA on $x$ with $p$, $x=pq+b$ for some $q,b\in\mathbb{Z}$ and $0\leq b \leq p-1$. Substitution for $x$,

\begin{align*}
a(pq+b)+py&=1, \\
apq + ab + py &= 1, \\
ab-1&= -py-apq, \\
ab-1&=p(-y-aq).
\end{align*}

\noindent By CPI, let $-y-aq=q'\in\mathbb{Z}$ such that $ab-1=pq'$. Thus, by definition, $ab\equiv 1\pmod p$. Furthermore, let $b'\in\mathbb{Z}$ where $1\leq b' < p$, and WLOG $b'<b$, such that $ab\equiv 1\pmod p$ and $ab'\equiv 1\pmod p$. By transitivity, $ab\equiv ab'\pmod p$. By definition,

\begin{align*}
ab-ab' &= pk
\end{align*}

\noindent for some $k\in\mathbb{Z}$. We find,

\begin{align*}
a(b-b')&=pk.
\end{align*}

\noindent This implies $p|a(b-b')$. Since $p\not|a$, $p|b-b'$. Since $1\leq b,b' <p$ implies $0\leq b-b' < p$, $p$ can only divide if $b-b'=0$. Thus, $b$ and $b'$ are the same, i.e., $b=b'$. Therefore, there exists a unique integer $b < p$ such that $ab\equiv 1\pmod p$.\qed

\end{document}

