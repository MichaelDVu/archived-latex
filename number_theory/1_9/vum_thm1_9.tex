\documentclass[12pt]{article}
\usepackage{fancyhdr}
\usepackage{amsmath}
\usepackage{amssymb, amsmath, graphicx,amsthm,setspace}
\pagestyle{fancy}

\lhead[lh-even]{\textbf{Michael Vu}}  
\lfoot[lf-even]{} 
\chead[ch-even]{\textbf{1.9 Theorem}}  
\cfoot[cf-even]{} 
\rhead[rh-even]{August 30, 2017}  
\rfoot[rf-even]{}

\begin{document}
\noindent\textbf{Lemma 0.} Let $n\in\mathbb{Z}$. Then $n\mid 0$.

\bigskip

\noindent\textbf{Proof.} Let $n\in\mathbb{Z}$ be given. By definition of divisibility, $0=0n$. Therefore, $n\mid 0$.

\bigskip

\noindent\textbf{1.9 Theorem.} Let $a,n\in\mathbb{Z}$ with $n>0$. Then $a\equiv a\pmod{n}$.

\bigskip

\noindent\textbf{Proof.} Let Let $a,n\in\mathbb{Z}$ with $n>0$ be given. Since $a-a=0$ and any $n>0$ divides 0, by Lemma 0, $n\mid(a-a)$. Therefore, by definition, $a\equiv a\pmod{n}$.\qed


\end{document}