\documentclass[12pt]{article}
\usepackage{fancyhdr}
\usepackage{amsmath}
\usepackage{amssymb, amsmath, graphicx,amsthm,setspace}
\pagestyle{fancy}

\lhead[lh-even]{\textbf{Michael Vu}}  
\lfoot[lf-even]{} 
\chead[ch-even]{\textbf{1.32 Theorem}}  
\cfoot[cf-even]{} 
\rhead[rh-even]{September 14, 2017}  
\rfoot[rf-even]{}

\begin{document}
\noindent\textbf{1.32 Theorem.} Let $a,n,b,r,k\in\mathbb{Z}$. If $k\mid a$, $k\mid b$, and $a=nb+r$, then $k\mid r$.

\bigskip

\noindent\textbf{Proof.} Let $a,n,b,r,k\in\mathbb{Z}$ be given such that $k\mid a$, $k\mid b$, and $a=nb+r$. Let $x,y\in\mathbb{Z}$ such that, by definition, $a=kx$ and $b=ky$. Substituting $a=nb+r$ into $a=kx$

\begin{align*}
a &= kx, \\
nb+r &= kx.
\end{align*}

\noindent Substituting for $b=ky$

\begin{align*}
n(ky)+r &= kx, \\
r &= kx-kny, \\
&= k(x-ny).
\end{align*}

By CPI, we can see $(x-ny)\in\mathbb{Z}$. Thus $k\mid r$.\qed


\end{document}