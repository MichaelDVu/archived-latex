\documentclass[12pt]{article}
\usepackage{fancyhdr}
\usepackage{amsmath}
\usepackage{amssymb, amsmath, graphicx,amsthm,setspace}
\pagestyle{fancy}

\lhead[lh-even]{\textbf{Michael Vu}}  
\lfoot[lf-even]{} 
\chead[ch-even]{\textbf{2.38 Theorem}}  
\cfoot[cf-even]{} 
\rhead[rh-even]{October 21, 2017}  
\rfoot[rf-even]{}

\begin{document}
\noindent\textbf{2.38 Theorem.} (Infinitude of $4k+3$ Primes Theorem) There are infinitely many prime numbers that are congruent to 3 modulo 4.

\bigskip

\noindent\textbf{Proof.} Suppose not. That is, suppose there is a finite set $S=\{p_1,p_2,...,p_m\}$ such that each $p_i$ is congruent to 3 modulo 4. Let a natural number (not in $S$) $n=2p_1p_2...p_m+1$. We know this number to be odd (by letting $p_1p_2...p_m=k$ such that $2k+1$) and no $p_i$ divides $n$. Thus, $n\equiv 1\pmod 4$ or $n\equiv 3\pmod 4$. Suppose $n\equiv 1\pmod 4$. By definition, $4k=n-1$. Substituting for $n$,

\begin{align*}
4k &= (2p_1p_2...p_m+1)-1 \\
&= 2p_1p_2...p_m.
\end{align*}

\noindent This implies $4|2p_1p_2...p_m$, which is not true since $p_1p_2...p_m$ is odd. Thus, $n\equiv 3\pmod 4$. By FTA, $n$ has a prime factorization such that $n=r_1r_2...r_t$. We want to show that one of these prime factors are congruent to 3 modulo 4 (there may be more, but we only need to show one) and that it is not in $S$. Since $n\equiv 3\pmod 4$, we can write $r_1r_2...r_m\equiv 3\pmod 4$. By Theorem 2.37, we know that one of these prime factors is not congruent 1 modulo 4, else $n\equiv 1\pmod 4$. Thus, at least one of these primes must be congruent 3 modulo 4. This contradicts our initial assumption because none of $n$'s primes are in set $S$. Therefore, there are infinitely many prime numbers that are congruent to 3 modulo 4.\qed

\end{document}