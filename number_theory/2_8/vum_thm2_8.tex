\documentclass[12pt]{article}
\usepackage{fancyhdr}
\usepackage{amsmath}
\usepackage{amssymb, amsmath, graphicx,amsthm,setspace}
\pagestyle{fancy}

\lhead[lh-even]{\textbf{Michael Vu}}  
\lfoot[lf-even]{} 
\chead[ch-even]{\textbf{2.8 Lemma}}  
\cfoot[cf-even]{} 
\rhead[rh-even]{September 30, 2017}  
\rfoot[rf-even]{}

\begin{document}
\noindent\textbf{2.8 Lemma.} Let $p$ and $q_{1},q_{2},...,q_{n}$ all be primes and let $k$ be a natural number such that $pk=q_{1}q_{2}...q_{n}$. Then $p=q_{i}$ for some $i$.

\bigskip

\noindent\textbf{Proof.} Let $p$ and $q_{1},q_{2},...,q_{n}$ all be primes and let $k$ be a natural number such that $pk=q_{1}q_{2}...q_{n}$ be given. Consider the base case where $n=1$ such that

\begin{equation*}
pk=q_{1}.
\end{equation*}

\noindent Since $p$ is prime, $k=1$ by primality. Thus, $p=q_{1}$. By induction, suppose $pk=q_{1}q_{2}...q_{b}$ with $1<n<b$ such that $p=q_{i}$. We want to show when $pk=q_{1}q_{2}...q_{b}q_{b+1}$, $p=q_{i}$. Since $pk=q_{1}q_{2}...q_{b}q_{b+1}$, we can rewrite this as 

\begin{equation*}
p|q_{1}q_{2}...q_{b}q_{b+1}.
\end{equation*}

\noindent Let $a=q_{1}q_{2}...q_{b}$ for some $a\in\mathbb{Z}$ such that $p|aq_{b+1}$. Suppose $(p,a)=1$. By Theorem 1.41, since $p|aq_{b+1}$ and $(p,a)=1$, $p|q_{b+1}$. By definition, $pt=q_{b+1}$ for some $t\in\mathbb{Z}$. Since $p$ is prime, $t=1$ by primality. Thus, $p=q_{i}$. Since our base case and inductive hypothesis is true, the Lemma is true.\qed

\end{document}