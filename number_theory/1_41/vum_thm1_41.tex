\documentclass[12pt]{article}
\usepackage{fancyhdr}
\usepackage{amsmath}
\usepackage{amssymb, amsmath, graphicx,amsthm,setspace}
\pagestyle{fancy}

\lhead[lh-even]{\textbf{Michael Vu}}  
\lfoot[lf-even]{} 
\chead[ch-even]{\textbf{1.41 Theorem}}  
\cfoot[cf-even]{} 
\rhead[rh-even]{September 19, 2017}  
\rfoot[rf-even]{}

\begin{document}
\noindent\textbf{1.41 Theorem.} Let $a,b,c\in\mathbb{Z}$. If $a\mid bc$ and $(a,b)=1$, then $a\mid c$.

\bigskip

\noindent\textbf{Proof.} Let $a,b,c\in\mathbb{Z}$ be given such that $a\mid bc$ and $(a,b)=1$. By Theorem 1.38, we know there exists $x,y\in\mathbb{Z}$ such that $ax+by=1$. Multiplying this equation by $c$,

\begin{align*}
c(ax+by) &= c(1), \\
acx + bcy &= c.
\end{align*}

\noindent Observing the left hand side of the equation, $a$ divides any integer multiple of $a$ and $a$ divides any integer multiple of $bc$. Since $a$ divides both terms, $a\mid c$.\qed

\end{document}