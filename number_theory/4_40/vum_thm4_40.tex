\documentclass[12pt]{article}
\usepackage{fancyhdr}
\usepackage{amsmath}
\usepackage{amssymb, amsmath, graphicx,amsthm,setspace}
\pagestyle{fancy}

\lhead[lh-even]{\textbf{Michael Vu}}  
\lfoot[lf-even]{} 
\chead[ch-even]{\textbf{4.40 Theorem}}  
\cfoot[cf-even]{} 
\rhead[rh-even]{November 28, 2017}  
\rfoot[rf-even]{}

\begin{document}
\noindent\textbf{4.40 Theorem.} If $p$ is a prime larger than 2, then $2\cdot 3\cdot 4\cdot \cdots \cdot (p-2)\equiv 1\pmod p$.

\bigskip

\noindent\textbf{Proof.} Let a prime $p>2$ be given. Notice for any integers $2\leq a,b \leq p-2$, there are $\frac{p-3}{2}$ distinct pairs such that $a\not= b$. By Theorem 4.36,

\begin{align*}
a_1b_1 &\equiv 1\pmod p,\\
a_2b_2 &\equiv 1\pmod p,\\
&\vdots \\
a_{\frac{p-3}{2}}b_{\frac{p-3}{2}} &\equiv 1\pmod p.
\end{align*}

\noindent By Theorem 4.38, $(a_1b_1)(a_2b_2)...(a_{\frac{p-3}{2}}b_{\frac{p-3}{2}})\equiv 1\pmod p$. Since these are distinct pairs such that $a\not= b$, $2\cdot 3\cdot 4\cdot \cdots \cdot (p-2)\equiv 1\pmod p$, provided $p>2$.\qed

\end{document}

