\documentclass[12pt]{article}
\usepackage{fancyhdr}
\usepackage{amsmath}
\usepackage{amssymb, amsmath, graphicx,amsthm,setspace}
\pagestyle{fancy}

\lhead[lh-even]{\textbf{Michael Vu}}  
\lfoot[lf-even]{} 
\chead[ch-even]{\textbf{4.9 Theorem}}  
\cfoot[cf-even]{} 
\rhead[rh-even]{November 9, 2017}  
\rfoot[rf-even]{}

\begin{document}
\noindent\textbf{4.9 Theorem.} Let $a,n\in\mathbb{N}$ with $(a,n)=1$ and let $k=\text{ord}_n(a)$. For any natural number $m$, $a^m$ is congruent modulo $n$ to one of the numbers $a^1,a^2,...,a^k$. 

\bigskip

\noindent\textbf{Proof.} By TDA, let $m-qk+r$ where $0\leq r\leq k-1$. Since $k=\text{ord}_n(a)$, $a^k\equiv 1\pmod n$. By Theorem 1.18,

\begin{align*}
(a^k)^q &\equiv 1^q\pmod n,\\
a^{qk} &\equiv 1\pmod n,\\
a^{qk}a^r &\equiv a^r\pmod n,\\
a^{qk+r} &\equiv a^r\pmod n,\\
a^m &\equiv a^r\pmod n.
\end{align*}

\noindent Recall $0\leq r\leq k-1$. If $r=0$, $a^m\equiv a^0\pmod n$ which is $a^m\equiv 1\pmod n$ and we find $a^m\equiv a^k\pmod n$. Thus, for any natural number $m$, $a^m$ is congruent modulo $n$ to one of the numbers $a^1,a^2,...,a^k$.\qed

\end{document}

