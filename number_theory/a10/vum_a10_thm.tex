\documentclass[12pt]{article}
\usepackage{fancyhdr}
\usepackage{amsmath}
\usepackage{amssymb, amsmath, graphicx,amsthm,setspace}
\pagestyle{fancy}

\lhead[lh-even]{\textbf{Michael Vu}}  
\lfoot[lf-even]{} 
\chead[ch-even]{\textbf{A.10 Theorem}}  
\cfoot[cf-even]{} 
\rhead[rh-even]{September 2, 2017}  
\rfoot[rf-even]{}

\begin{document}
\noindent\textbf{A.10 Theorem.} Let $n$ be a natural number. Then $1+2+3+\cdots +n= \frac{n(n+1)}{2}$.

\bigskip

\noindent\textbf{Proof.} Let $P(n)$ be the statement $1+2+3+\cdots+n=\frac{n(n+1)}{2}$ and $n\in\mathbb{N}$. We consider the base case where $n=1$. 

\bigskip

\noindent For $P(1)$,

\begin{equation*} 
\begin{align*}
1 &= \frac{1(1+1)}{2} \\
&= \frac{2}{2} \\
&= 1.
\end{align*}
\end{equation*}

\noindent Since the base case is true, we will prove by induction. Suppose now $1+2+3+\cdots+k=\frac{k(k+1)}{2}$ for some $k\in\mathbb{N}$. We want to show $1+2+3+\cdots+k+(k+1)=\frac{(k+1)(k+2)}{2}$. It follows,

\begin{equation*} 
\begin{align*}
1+2+3+\cdots+k+(k+1) &= \frac{k(k+1)}{2}+(k+1) \\
&= \frac{k(k+1)}{2}+\frac{2(k+1)}{2} \\
&= \frac{k(k+1)+2(k+1)}{2} \\
%&= \frac{k^2+k+2k+2}{2} \\
%&= \frac{k^2+3k+2}{2} \\
&= \frac{(k+1)(k+2)}{2}.
\end{align*}
\end{equation*}

\noindent Thus, by induction, $1+2+3+\cdots +n= \frac{n(n+1)}{2}$.\qed

\end{document}