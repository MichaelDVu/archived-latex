\documentclass[12pt]{article}
\usepackage{fancyhdr}
\usepackage{amsmath}
\usepackage{amssymb, amsmath, graphicx,amsthm,setspace}
\pagestyle{fancy}

\lhead[lh-even]{\textbf{Michael Vu}}  
\lfoot[lf-even]{} 
\chead[ch-even]{\textbf{3.29 Theorem}}  
\cfoot[cf-even]{} 
\rhead[rh-even]{November 4, 2017}  
\rfoot[rf-even]{}

\begin{document}
\noindent\textbf{Lemma.} Let $n_1,n_2,...,n_k$ be natural numbers. If $(n_i,n_j)=1$ such that $i\not= j$ and $1 \leq i,j, \leq k$, then $(n_1n_2...n_{k-1},n_k)=1$.

\bigskip

\noindent\textbf{Proof.} Let $k=3$ be our base case. By Theorem 2.29, $(n_1n_2,n_3)=1$. Suppose all $k$ is true where $1 \leq k \leq t$. We want to show $(n_1n_2...n_{t},n_{t+1})=1$. Since we know up to $t$ is true, by Theorem 2.29, $(n_1n_2...n_{t},n_{t+1})=1$. Thus, if $(n_i,n_j)=1$ such that $i\not= j$ and $1 \leq i,j, \leq k$, then $(n_1n_2...n_{k-1},n_k)=1$.\qed

\bigskip

\noindent\textbf{3.29 Theorem.} (Chinese Remainder Theorem). Suppose $n_1,n_2,...,n_L$ are positive integers that are pairwise relatively prime, that is, $(n_i,n_j)=1$ for $i\not= j$, $1\leq i,j \leq L$. Then the system of $L$ congruences

\begin{align*}
x &\equiv a_1\pmod {n_1} \\
x &\equiv a_2\pmod {n_2} \\
\vdots \\
x &\equiv a_L\pmod {n_L} \\
\end{align*}

\noindent has a unique solution modulo the product $n_1n_2...n_L$.

\bigskip

\noindent\textbf{Proof.} Let $L=2$. Consider this the base case. Thus,

\begin{align*}
x &\equiv a_1 \pmod{n_1}\\
x &\equiv a_2 \pmod{n_2}.
\end{align*}

\noindent Since $(n_1,n_2)=1$, by Theorem 3.28, $x\equiv x' \pmod{n_1n_2}$. Thus, the base case is true. Suppose this is true for all $L$ where $1\leq L \leq K$. By induction, we want to show 

\begin{align*}
x &\equiv a_1\pmod {n_1} \\
x &\equiv a_2\pmod {n_2} \\
\vdots \\
x &\equiv a_L\pmod {n_K} \\
x &\equiv a_L\pmod {n_{K+1}}
\end{align*}

\noindent also has a unique solution modulo the product $n_1n_2...n_Kn_{K+1}$. Thus, the system of congruences is

\begin{align*}
x &\equiv a_1\pmod {n_1} \\
x &\equiv a_2\pmod {n_2} \\
\vdots \\
x &\equiv a_K \pmod {n_K} \\
x &\equiv a_{K+1}\pmod {n_{K+1}}.
\end{align*}

\noindent By our induction hypothesis and Theorem 3.28, we know up to $K$ is \\
$x\equiv x' \pmod{n_1n_2...n_K}$. Thus,

\begin{align*}
x &\equiv x' \pmod{n_1n_2...n_K}\\
x &\equiv a_{K+1}\pmod {n_{K+1}}.
\end{align*}

\noindent   By Theorem 3.28, since $(n_1n_2...n_K,n_{K+1})=1$ by the Lemma and Theorem 2.29, and solution $x$ satisfies

\begin{equation*}
x \equiv x'' \pmod{n_1n_2...n_Kn_{K+1}},
\end{equation*}

\noindent for $x''\in\mathbb{Z}$. Thus, the system of $L$ congruences has a unique solution modulo the product $n_1n_2...n_L$.\qed

\end{document}

