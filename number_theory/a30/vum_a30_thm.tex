\documentclass[12pt]{article}
\usepackage{fancyhdr}
\usepackage{amsmath}
\usepackage{amssymb, amsmath, graphicx,amsthm,setspace}
\pagestyle{fancy}

\lhead[lh-even]{\textbf{Michael Vu}}  
\lfoot[lf-even]{} 
\chead[ch-even]{\textbf{A.30 Theorem}}  
\cfoot[cf-even]{} 
\rhead[rh-even]{September 5, 2017}  
\rfoot[rf-even]{}

\begin{document}
\noindent\textbf{A.30 Theorem.} Every natural number can be written as the sum of distinct powers of 2.

\bigskip

\noindent\textbf{Proof.} Assume $P(k)$ for $n\leq k$ can be written as the sum of distinct powers of 2. We will prove $P(n)$ for all natural numbers $n$ by strong induction. 

\bigskip

\noindent\textbf{Base Case.} For $P(1)$,

\begin{align*}
1 &= 2^0 \\
&= 1.
\end{align*}

\noindent The base case $P(1)$ is true.

\bigskip

\noindent\textbf{Inductive Step.} Suppose $P(n)$ is true for every natural number $n\leq k$. We want to show $P(k+1)$ is also true.

\bigskip

\noindent\textbf{Case 1.} If $k$ is even, it is easy to see that the sum of distinct power of 2 does not contain $2^0$. When we consider $k+1$, it is easy to see this is the previous sum plus $2^0=1$, a distinct power of 2.

\bigskip

\noindent\textbf{Case 2.} If $k$ is odd, then $k=2m-1$ for some $m\in\mathbb{Z}$. Then $k+1=2m$. Since $m\leq k$, $m$ can be written as the sum of distinct powers of 2, the same way $k+1$ is true in the previous case.

\bigskip

\noindent Since the base case of $P(1)$ and both cases of $P(k+1)$ are true, every natural number can be written as the sum of distinct powers of 2.\qed

\end{document}