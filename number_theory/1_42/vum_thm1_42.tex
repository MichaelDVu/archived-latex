\documentclass[12pt]{article}
\usepackage{fancyhdr}
\usepackage{amsmath}
\usepackage{amssymb, amsmath, graphicx,amsthm,setspace}
\pagestyle{fancy}

\lhead[lh-even]{\textbf{Michael Vu}}  
\lfoot[lf-even]{} 
\chead[ch-even]{\textbf{1.42 Theorem}}  
\cfoot[cf-even]{} 
\rhead[rh-even]{September 24, 2017}  
\rfoot[rf-even]{}

\begin{document}
\noindent\textbf{1.42 Theorem.} Let $a,b,n\in\mathbb{Z}$. If $a\mid n$, $b\mid n$ and $(a,b)=1$, then $ab\mid n$.

\bigskip

\noindent\textbf{Proof.} Let $a,b,n\in\mathbb{Z}$ be given such that $a\mid n$, $b\mid n$, and $(a,b)=1$. By definition, $a\mid n$ and $b\mid n$ are equivalent to $n=as$ and $n=bt$, respectively, for some $s,t\in\mathbb{Z}$. By Theorem 1.38, since $(a,b)=1$, there exists $x,y\in\mathbb{Z}$ such that $ax+by=1$. Multiplying both sides by $n$,

\begin{align*}
n &= n(ax+by) \\
&= nax + nby.
\end{align*}

\noindent Replacing $n$ in the right hand side,

\begin{align*}
n &= bt(ax) + as(by) \\
&= ab(xt) + ab(sy)\\
&= ab(xt+sy).
\end{align*}

\noindent By CPI, $xt+sy$ is an integer. Thus, $ab\mid n$.\qed

\end{document}