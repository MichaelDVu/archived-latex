\documentclass[12pt]{article}
\usepackage{fancyhdr}
\usepackage{amsmath}
\usepackage{amssymb, amsmath, graphicx,amsthm,setspace}
\pagestyle{fancy}

\lhead[lh-even]{\textbf{Michael Vu}}  
\lfoot[lf-even]{} 
\chead[ch-even]{\textbf{A.20 Theorem}}  
\cfoot[cf-even]{} 
\rhead[rh-even]{September 5, 2017}  
\rfoot[rf-even]{}

\begin{document}
\noindent\textbf{A.20 Theorem.} For every natural number $n>3$, $2^n < n!$.

\bigskip

\noindent\textbf{Proof.} Let $P(n)$ be the statement $2^n < n!$ for$n>3$. We consider the base case where $n=4$. 

\bigskip

\noindent For $P(4)$,


\begin{align*}
2^4 &< 4!, \\
16 &< 24.
\end{align*}


\noindent Since the base case is true, we will prove by induction. Suppose now, $2^k < k!$ for $k>3$. We want to show $2^{k+1} < (k+1)!$. Multiplying both sides by $k+1$,


\begin{align*}
(k+1)2^k &< (k+1)k!.
\end{align*}


\noindent Observing the right side of (1) and remembering algebra, for any natural number $z$, $z!=z(z-1)!$. Substituting $z$ for $k+1$,

 
\begin{align*}
z(z-1)! &= (k+1)(k+1-1)! \\
&= (k+1)k! \\
&= (k+1)!.
\end{align*}


\noindent Now we find (1) to be 

\begin{equation} 
(k+1)2^k < (k+1)!. 
\end{equation}

\noindent Suppose we compare the left side of (2) to the left side of the hypothesis. Assuming $k\geq 4$, we can say $2<(k+1)$ is true. Factoring, we find $2^{k+1}=(2)2^k$. Because $2<(k+1)$, we can say


\begin{align*}
2^{k+1} &< (k+1)2^k < (k+1)!, \\
2^{k+1} &< (k+1)!.
\end{align*}


\noindent Since $P(k+1)$ is true, given $P(k)$ is true, and the base case of $P(n=4)$ is true, $2^n < n!$ for every natural number $n>3$ is true by induction.\qed

\end{document}