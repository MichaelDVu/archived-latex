\documentclass[12pt]{article}
\usepackage{fancyhdr}
\usepackage{amsmath}
\usepackage{amssymb, amsmath, graphicx,amsthm,setspace}
\pagestyle{fancy}

\lhead[lh-even]{\textbf{Michael Vu}}  
\lfoot[lf-even]{} 
\chead[ch-even]{\textbf{3.16 Theorem}}  
\cfoot[cf-even]{} 
\rhead[rh-even]{October 24, 2017}  
\rfoot[rf-even]{}

\begin{document}
\noindent\textbf{3.16 Theorem.} Let $n$ be a natural number. Every complete residue system modulo $n$ contains $n$ elements.

\bigskip

\noindent\textbf{Proof.} Suppose not. That is, let the set $\{a_1,a_2,...,a_k\}$ of integers not contain $n$ elements. Suppose this set has less than $n$ elements, then by definition of CRS, one element from the integer set is congruent to two or more CCRS modulo $n$, which is a contradiction to the definition. On the otherhand, letting the set have more than $n$ elements implies two or more $a_i$ are congruent to one of the CCRS modulo $n$, which also contradicts the definition of CRS. Thus, every complete residue system modulo $n$ contains $n$ elements.\qed

\end{document}

