\documentclass[12pt]{article}
\usepackage{fancyhdr}
\usepackage{amsmath}
\usepackage{amssymb, amsmath, graphicx,amsthm,setspace}
\pagestyle{fancy}

\lhead[lh-even]{\textbf{Michael Vu}}  
\lfoot[lf-even]{} 
\chead[ch-even]{\textbf{2.32 Theorem}}  
\cfoot[cf-even]{} 
\rhead[rh-even]{October 7, 2017}  
\rfoot[rf-even]{}

\begin{document}
\noindent\textbf{2.32 Theorem.} For all natural numbers $n$, $(n,n+1)=1$. 

\bigskip

\noindent\textbf{Proof.} Let $n\in\mathbb{N}$ be given. Let $(n,n+1)=d$ with $d\geq 1$ such that $d|n$ and $d|(n+1)$. By Theorem 1.2, $d|[n-(n+1)]$. By definition, $[n-(n+1)]=dt$ for $t\in\mathbb{Z}$. Thus,

\begin{align*}
dt &= n-n+1 \\
&= 1.
\end{align*}

\noindent Since $d\geq 1$, we could assume $d=1$. But suppose not. That is, given $dt=1$, suppose $d>1$ such that $1\leq t < dt$. It follows that $1<dt$. This is a contradiction to our given $dt=1$. Thus, $d=1=(n,n+1)$.\qed

\end{document}