\documentclass[12pt]{article}
\usepackage{fancyhdr}
\usepackage{amsmath}
\usepackage{amssymb, amsmath, graphicx,amsthm,setspace}
\pagestyle{fancy}

\lhead[lh-even]{\textbf{Michael Vu}}  
\lfoot[lf-even]{} 
\chead[ch-even]{\textbf{4.10 Theorem}}  
\cfoot[cf-even]{} 
\rhead[rh-even]{November 14, 2017}  
\rfoot[rf-even]{}

\begin{document}
\noindent\textbf{4.10 Theorem.} Let $a,n\in\mathbb{N}$ with $(a,n)=1$ and let $k=\text{ord}_n(a)$, and let $m\in\mathbb{N}$. Then $a^m\equiv 1\pmod n$ if and only if $k|m$.

\bigskip

\noindent\textbf{Proof.} Suppose $k|m$. By definition, $m=kt$ for $t\in\mathbb{Z}$. Since $k=\text{ord}_n(a)$,

\begin{align*}
a^k &\equiv 1\pmod n,\\
(a^k)^t &\equiv 1^t\pmod n,\\
a^{kt} &\equiv 1\pmod n.
\end{align*}

\noindent Thus, $a^m\equiv 1\pmod n$. Now suppose $a^m\equiv 1\pmod n$. Applying TDA, $m=kq+r$ for some $q,r\in\mathbb{Z}$ with $0\leq r \leq m-1$. Since $k=\text{ord}_n(a)$,

\begin{align*}
a^k &\equiv 1\pmod n,\\
(a^k)^q &\equiv 1^q\pmod n,\\
a^{kq} &\equiv 1\pmod n\\
a^{kq}a^r &\equiv a^r\pmod n\\
a^{kq+r} &\equiv a^r\pmod n\\
a^m &\equiv a^r\pmod n.
\end{align*}

\noindent Since we have defined $0\leq r\leq k-1$, if $r\geq 1$, then $a^r\equiv 1\pmod n$ contradicts $k=\text{ord}_n(a)$. Thus, $r=0$ such that $a^{m} \equiv a^r \equiv a^0 \equiv 1\pmod n$. This implies that $m=kq$, and by definition, $k|m$. Thus, $a^m\equiv 1\pmod n$ if and only if $k|m$, provided $(a,n)=1$ and $k=\text{ord}_n(a)$.\qed

\end{document}

